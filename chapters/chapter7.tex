\chapter{Conclusions and future work}
\label{chap:conclusions_future_work}
\lhead{Chapter 7. \emph{Conclusions and future work}}
In this work we have done a systematic compilation of theoretical and practical aspects of Lubrication Theory. These studies have involved Fluid Mechanics, Lubrication Theory, Elliptic PDE's from Functional Analysis, Calculus of Variations, and Numerical Analysis from varied aspects like numerical convergence, stability, simulations setup and results interpretation. Nowadays, the organization made here cannot be found in other publications. Therefore, this document could be used by undergraduate/graduate students who want to have a global understanding of the theoretical and practical subjects of the theory addressing modeling and simulation of lubricated contacts. Next, we present particular conclusions from the contents.

We have seen in \Chapref{equations_lubrication} that Reynolds equation for the stationary case is satisfied by the limit solutions of the Stokes equations when the proximity parameter $H/L$ goes to zero. This interesting point of view gives more insight into the nature of Reynolds equation when compared to just dropping the terms of order $H/L$ (or higher), as is done in \Secref{lub_hyp_in_nvs}, \nameref{sec:lub_hyp_in_nvs}.

When considering discontinuities in the gap function $h$, the mathematical tools needed rely on the theory of PDE's from the Functional Analysis Theory, as it was shown in \Chapref{maths_reynolds_equation}. In this way, it is possible not just to study the well-posedness of Reynolds equation in the presence of discontinuities but also to give a first approach to a cavitation model by imposing an obstacle to the pressure solution (see \Secref{reynolds_model}, \nameref{sec:reynolds_model}).

The Reynolds cavitation model does not enforce mass-conservation. This key issue can be tackled with the Elrod-Adams cavitation model as presented in \Chapref{cavitation_models}. The Elrod-Adams model introduces a change in the type of PDE that models the fluid's behavior. The analysis of the problem with that change in the PDE type is hard to do, in such a way that nowadays only an existence result is available.

In \Chapref{slider_bearing} we have shown that the models and resolution algorithms presented and studied along the previous chapters of this document can be used effectively for the simulation of one-dimensional slider bearings. It is observed that these results are in line with the literature. Moreover, interesting phenomena can be revealed, such as the catastrophic event depending on the collapse of the cavitation bubbles, and the hysteresis of the stationary state. Also, we have seen how the convergent-divergent geometries of the sinusoidal pockets plays different roles depending on the size of the pressurized zone. Nevertheless, we must emphasize that such discoveries depend upon the models used, as neither the Reynolds model nor the Elrod-Adams model are free of criticism (see for instance \cite{organisciak2007phd,buscaglia13}).

Next, we present different possibilities to continue the present work.

\subsection*{Limit equations}
In \Chapref{maths_reynolds_equation} we discussed the work of \citeauthor{chambat1986} \cite{chambat1986} where it is rigorously proved that Reynolds equation is fulfilled by the limit solutions (in the sense of the proximity of the surfaces) of the Stokes system. However, the hypothesis made on the regularity of the domain, being of class $C^1$, is very strong. Also, the gap function $h$ is supposed to be continuously differentiable over the closure of the domain, a hypothesis that is not compatible with real applications as they might involve discontinuous surfaces. It would be interesting to study the possibility of extending this work for weaker regularity hypothesis and, moreover, seek for a similar result for the limit formulas of friction, which were found in \Chapref{equations_lubrication} by making use of an asymptotic limit.

\subsection*{Elrod-Adams extensions}
The Elrod-Adams model, as discussed by \citeauthor{buscaglia13} \cite{buscaglia13} and also by \citeauthor{checophd} \cite{checophd}, assume the transport velocity of the fluid in the cavitated zone to be $S/2$, i.e., equal to the transport velocity that is obtained when the fluid film is complete. This assumption is not realistic enough for the cases considered in this work, in which most of the lubricant lies on the moving surface (the runner). To extend Elrod-Adams for allowing transport greater than $S/2$ is a challenging problem, lack of uniqueness of solution is one of its difficulties; and an algorithm that automatically keeps track of the cavitation boundary (so-called front-capturing algorithms, e.g., Elrod-Adams algorithm) is not available yet (some efforts on this issue can be found in \cite{checophd}).

\subsection*{Discontinuous Galerkin}

Currently the adopted method has poor mesh flexibility, since it is based on a rectangular grid. Its convergence order is also low, as shown in \Secref{num_impl_reyolds_cav_models} \nameref{sec:num_impl_reyolds_cav_models}, because it assumes a piecewise constant interpolation of the variables. Standard high order methods cannot be applied because of the spontaneously-generated discontinuities at cavitation boundaries. With this perspective, it would be interesting to set focus on Discontinuous Galerkin (DG) Methods, which overcome these issues for elliptic problems \cite{cb2013}, hyperbolic problems \cite{cockburn2001,cockburn2003},  and elastohydrodynamic lubrication problems \cite{lbgj2012}.

A deep study of DG methods can be found in \cite{abcm2002}. The change of type of PDE, from elliptic to hyperbolic at the cavitation boundaries, makes it an interesting application of DG approximation techniques. Further, the generalized cavitation model developed by Buscaglia et al \cite{buscaglia13,ausas2013}, which allows to vary the transport velocity, requires the imposition of inflow boundary conditions at the cavitation boundary. This would certainly require some specific development of DG techniques.

\subsection*{Boundary conditions for pressure}
The current cavitation models only admit a constant cavitation pressure, and it must be equal to the surrounding pressure. This has been questioned by several researchers \cite{shen2013} and is a source of inaccuracy. At some instants in the engine cycle the pressure difference between both sides of the ring pack can reach 100 [atm]. Therefore, it is interesting to include this consideration when developing new cavitation models.