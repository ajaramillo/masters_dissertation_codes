\chapter{Mathematics of Reynolds equation}
\label{chap:maths_reynolds_equation} % For referencing the chapter elsewhere, use \ref{Chapter1} 
\lhead{Chapter 3. \emph{Mathematics of Reynolds equation}} % This is for the header on each page - perhaps a shortened title

We have shown in \Chapref{equations_lubrication} that Reynolds equation models the tribological variables of two surfaces being lubricated. In this  chapter a mathematical analysis is developed in order to study the \emph{well-posedness} of Reynolds equation. Using powerful tools of Functional Analysis, like the Hilbert Spaces structure, \emph{existence}, \emph{uniqueness} and \emph{stability} of solutions of Reynolds equation will be addressed. Furthermore, we will seek for \emph{regularity} of the solutions, i.e., how much smooth the solutions are. As the reader may guess, the last question will be related to the quality of the input: how \emph{regular} is the gap between the surfaces?; how \emph{regular} is the boundary of the domain?.

For this we will consider a measurable domain $\Omega\subset\mathbb{R}^2$, with Lebesgue measure \break $\mu(\Omega)<+\infty$, and a measurable subdomain $\omega\subset  \Omega$ where Reynolds equation holds. In this chapter, $\omega$ is a data of the problem and it is supposed to be locally Lipschitz (see definition \ref{def:local_lipschitz}). The general problem, where $\omega$ is also an unknown, will be studied in \Chapref{cavitation_models} where $\Omega \setminus \omega$ will be determined by the cavitation phenomenon.

\section{From Stokes equations to Reynolds equation}
Along \Secref{lub_hyp_in_nvs,reynolds_equation} we have made asymptotic expansions for obtaining Reynolds equation from Navier-Stokes equations. \citeauthor{chambat1986} (\citeyear{chambat1986}) \cite{chambat1986} proved mathematically that Reynolds equation is an approximation of Stokes equations. In the following, we summarize their results in order to give a mathematical comprehension of the relation between both sets of equations. 

Consider two surfaces in proximity and in relative motion (see \Figref{bayadachambat_scheme}). The first surface (lower one), denoted by $\omega$, is a planar bounded domain of $\mathbb{R}^2$ placed in the plane $z=0$ and its boundary $\partial \omega$ is locally Lipschitz. The second surface (upper one) is characterized by $z=H(x,y)$, $(x,y)\in \omega$. The thin distance between both surfaces is taken into account by introducing a small parameter $\epsilon$, which will tend to $0$, and a fixed function $h:\omega\rightarrow \mathbb{R}^+$ such that $$H(x,y)=\epsilon \,h(x,y),$$
with $h\in C^1(\overline{\omega})$ and $h\geq \alpha >0$. 
%\begin{equation}
%\frac{\partial }{\partial x}\left(H^3\frac{\partial p}{\partial x}\right)+\frac{\partial }{\partial y}\left(H^3\frac{\partial p}{\partial y}\right)=6\,S\frac{\partial H}{\partial x}.\label{eq:reynolds_equation_stokescomparison}
%\end{equation}
 \begin{figure}[h]
 \centering 
 \def\svgwidth{\textwidth}	
\input{figs/bayadachambat_scheme.pdf_tex}
\caption[Domain dependent on $\epsilon$ for studying the convergence of Stokes system solutions.]{$\Omega_\epsilon$ scheme. Based on Fig. 1 in \cite{chambat1986}.}\label{fig:bayadachambat_scheme}
\end{figure}

Let us write the domain $$\Omega_\epsilon=\{(x,y,z)\in \mathbb{R}^3,\,(x,y)\in \omega, \,0 < z<H(x,y)\},$$and $\Gamma_\epsilon=\partial \Omega_\epsilon=\overline{\omega}\,\cup\,\overline{\Gamma}_1^\epsilon\,\cup\,\overline{\Gamma}_L^\epsilon$ its boundary (see \Figref{bayadachambat_scheme}). On $\Omega_\epsilon$, the Stokes system\footnote{Assuming no source term on the right hand side of \Eqref{stokes_system} as generally occurs in Lubrication Theory.} and the continuity equation for a Newtonian fluid can be written resp. as
\begin{align}
-\mu\nabla ^2\, \mathbf{U}^\epsilon+\nabla p^\epsilon&= 0 \label{eq:stokes_system}\\
\nabla \cdot \mathbf{U}^\epsilon &=0,\label{eq:stokes_continuity}
\end{align}
where $\mu$ is the dynamic viscosity, $\mathbf{U}^\epsilon$ is the velocity field of the fluid and $p^\epsilon$ is its hydrodynamic pressure. Dirichlet boundary conditions for the velocities $\mathbf{U}^\epsilon=(g^\epsilon,0,0)$ on $\Gamma^\epsilon$ are imposed, where
\begin{align}
g^\epsilon=0,\qquad \text{on }\Gamma_1^\epsilon\label{eq:cond_rey_b1}\\
g^\epsilon=S>0,\qquad \text{on }\omega.\label{eq:cond_rey_b2}
\end{align}
Also, in order to make sure that the Stokes equations have a solution, the authors \cite{chambat1986} impose the condition
\begin{equation}
g^\epsilon \in H^{1/2}(\Gamma^\epsilon)\qquad \text{and}\qquad \int_{\Gamma^\epsilon_L} g^\epsilon \cos(\bds{\hat{n}},\bds{\hat{e}}_1)\, d\sigma =0,\label{eq:cond_rey_b3}
\end{equation}
where $\mathbf{\hat{n}}$ is the normal unit vector pointing outward $\Omega_\epsilon$ and $\bds{\hat{e}}_1$ is the unit vector pointing positively along the $x$-axis. The first condition is a regularity requirement and the second condition is for mass-conservation.
\subsection*{Existence and uniqueness for Stokes system}
First, let us define the space $L_0^2(\Omega_\epsilon)=\{f\in L^2(\Omega_\epsilon):\int_{\Omega_\epsilon} f\,dV=0\}$, $dV=dx\,dy\,dz$, which is the class of functions with zero average. This set is considered since the pressure is uniquely determined up to an additive constant.

The next theorem establishes the existence and uniqueness of the Stokes problem defined by equations \eqref*{stokes_system}-\eqref*{cond_rey_b2}. It is a well known result and it can be found, for instance, in \cite{girault1981}:
 \begin{theorem}\label{theo:t1}
Under assumptions \eqref*{cond_rey_b1,cond_rey_b2,cond_rey_b3}, there exists a unique pair of functions $(\mathbf{U}^\epsilon,p^\epsilon)$ in $(H^1(\Omega_\epsilon))^3\times L_0^2(\Omega_\epsilon)$ such that \begin{align*}
-\mu\nabla ^2\, \mathbf{U}^\epsilon+\nabla p^\epsilon&= 0 \\
\nabla \cdot \mathbf{U}^\epsilon &=0\\
\mathbf{U}^\epsilon&=(g^\epsilon,0,0),\qquad\text{on }\Gamma^\epsilon.
\end{align*}
\end{theorem}
Moreover, let us define the bilinear form $a$ by $a(\mathbf{U},\mathbf{V})=\sum_{i=1}^3\int_{\Omega_\epsilon}\nabla  u_i \cdot\nabla v_i\, dV$. Then, $(\mathbf{U}^\epsilon,p^\epsilon)$ satisfies the weak formulation:
\begin{align*}
\mu \,a(\mathbf{U}^\epsilon,\Phi) &= \int_{\Omega_\epsilon}p^\epsilon\, \nabla\cdot \Phi\,dV&\forall \Phi \in (H^1_0(\Omega_\epsilon))^3\\
0&=\int_{\Omega_\epsilon} q\, \nabla\cdot \mathbf{U}^\epsilon\,dV,&\forall q\in L_0^2(\Omega_\epsilon),
\end{align*}
and there exists a function $\mathbf{G}^\epsilon \in H^1(\Omega_\epsilon)^3$ such that
\begin{equation}
\nabla \cdot \mathbf{G}^\epsilon =0,\qquad \mathbf{G}^\epsilon-\mathbf{U}^\epsilon\in (H^1_0(\Omega_\epsilon))^3.
\end{equation}
%With this, substituting the next asymptotic expansion (see Section ``Homogenization'' in \cite{evans1997})
%\begin{align*}
%p^\epsilon(x,x_3)&=\frac{1}{\epsilon^2}p^{-2}(x,x_3)+\frac{1}{\epsilon}p^{-1}(x,x_3)+p^0(x,x_3)+\ldots\\
%u_i^{\epsilon}(x,x_3)&=u_i^0(x,x_3)+\epsilon\, u_i^1(x,x_3)+\epsilon^2u_i^2(x,x_3)+\ldots\qquad i=1,2,3
%\end{align*}
%into the Stokes equations, it is obtained the Reynolds equation
%\begin{equation}
%\nabla\cdot(h^3\nabla p^{-2})=6S\frac{\partial h}{\partial x_1}.\label{eq:reynolds_from_stokes}
%\end{equation}
Now, set the domain $\Omega=\{(x,y,Z)\in \mathbb{R}^3,\,(x,y)\in \omega,\, 0 < Z<h(x,y)\}$, and for any function $v(x,y,z)$ defined on $\Omega_\epsilon$ associate the function $\hat{v}(x,y,Z)=v(x,y,\epsilon\,Z)$ defined on $\Omega$. 

Along this definitions and using Functional Analysis (e.g.\footnote{The reader can found a summary of the main results in Appendix \ref{chap:appendix_maths}.}, Chapter ``Banach and Hilbert Spaces'' in \cite{trudinger1983}, Chapters III and V in \cite{adams1975}) the authors \cite{chambat1986} obtained the next results regarding convergence of the functions $\mathbf{\hat{U}}^\epsilon$ and $\hat{p}^\epsilon$.
\subsection*{Convergence of the solutions}
\begin{theorem}\label{theo:t2}
Suppose there exists a constant $K$, not depending on $\epsilon$, such that $G^\epsilon$ in Theorem \ref{theo:t1} satisfies
\begin{equation}
\|\nabla \hat{G}_i^\epsilon\|_{\left(L^2(\Omega)\right)^3}\leq K,\qquad i=1,2,3,\label{eq:rest_G}
\end{equation}
then, there exists $\mathbf{U}^*$ in $(L^2( \Omega))^3$ such that
\begin{align*}
\mathbf{\hat{U}}^\epsilon \rightarrow \mathbf{U}^*,\qquad \frac{\partial \mathbf{\hat{U}}^\epsilon}{\partial Z} \rightarrow \frac{\partial \mathbf{U}^*}{\partial Z},\qquad \epsilon \frac{\partial \mathbf{\hat{U}^\epsilon}}{\partial x} \rightarrow 0,\qquad \epsilon \frac{\partial \mathbf{\hat{U}^\epsilon}}{\partial y} \rightarrow 0
\end{align*}
weakly in $(L^2( \Omega))^3$.
\end{theorem}
The proof of Theorem \ref{theo:t2} is based on the following estimates that are proved by the authors \cite{chambat1986} (under the hypothesis of Theorem \ref{theo:t2})
$$\|\mathbf{\hat{U}^\epsilon}\|_{\left(\lpOm{2}\right)^3}\leq K,\qquad \left\lVert\parder{\hat{u}_i^\epsilon}{\xi}\right\rVert_\lpOm{2}\leq \frac{K}{\epsilon},\qquad \left\lVert\parder{\hat{u}^\epsilon_i}{Z}\right\rVert_\lpOm{2}\leq K,\qquad i=1,2,3,\,\xi\in\{x,y\}.$$
Also a result on the convergence of $p_\epsilon$ is given, which is based on the next estimates
$$\left\lVert \parder{\hat{p}_\epsilon}{x}\right\rVert_{H^{-1}(\Omega)} \leq \frac{K}{\epsilon^2},\,\qquad \left\lVert \parder{\hat{p}_\epsilon}{y}\right\rVert_{H^{-1}(\Omega)} \leq \frac{K}{\epsilon^2},\qquad  \left\lVert \parder{\hat{p}_\epsilon}{Z}\right\rVert_{H^{-1}(\Omega)} \leq \frac{K}{\epsilon}.$$
\begin{theorem}
There exists $p^*$ in $L_0^2(\Omega)$ such that $\epsilon^2\hat{p}_\epsilon$ converges weakly to $p^*$; moreover $\parder{p^*}{Z}=0$.
\end{theorem}
\subsection*{Functional relations of the limit solutions}
Once the existence of limit solutions was established, the authors found that these limits accomplishes analogous equations as those found in \Secref{lub_hyp_in_nvs,reynolds_equation}.
%After this, it is found that these limit functions of $p^\epsilon$ and $\mathbf{\hat{U}}^\epsilon$  accomplishes equations analogous to those found in \Chapref{chapter2}, i.e. \eqref{pres_prof_x,pres_prof_y}.
\begin{theorem}
Under the same hypothesis of Theorem \ref{theo:t2}, the components of the limit field $\mathbf{U}^*$ satisfies the equations:
\begin{align*}
\frac{\partial p^*}{\partial x}&=\mu \frac{\partial^2 u_1^*}{\partial Z^2},&\text{in }&H^{-1}(\Omega)\\
\frac{\partial p^*}{\partial y}&=\mu\frac{\partial^2 u_2^*}{\partial Z^2},&\text{in }&H^{-1}(\Omega)\\
 u_3^*&=0,&\text{in }&\Omega.
\end{align*}
\end{theorem}
Now, for any function $v(x,y,z)$ in $H^1(\Omega_\epsilon)$, or the corresponding $\hat{v}(x,y,Z)$ in $H^1(\Omega)$, define the average $$\overline{v}(x,y)=\frac{1}{h}\int_0^h \hat{v}(x,y,Z)\,dZ=\frac{1}{\epsilon\,h}\int _0^{\epsilon\,h}v(x,y,z)\,dz$$
so $\overline{v}$ lies in $H^1(\omega)$.

\begin{theorem}
Under the same hypothesis of Theorem \ref{theo:t2}, the average velocity field $\bar{u}^*$ satisfies
\begin{align*}
\bar{u}_1^*=\frac{S}{2}-\frac{h^2}{12\mu}\frac{\partial p^*}{\partial x},\,\bar{u}_2^* ={}& -\frac{h^2}{12\mu}\frac{\partial p^*}{\partial y} && \text{both in }H^{-1}(\omega),\\
\bar{u}_3^* ={}& 0 & &\text{in }\omega.
\end{align*}
Moreover, the average velocity field $\overline{\mathbf{U}}^\epsilon$ satisfies the ``mass flow conservation'' equation:
\begin{equation*}
\frac{\partial}{\partial x}\left(h\,\overline{u}_1^\epsilon\right)+\frac{\partial}{\partial y}\left(h\,\overline{u}_2^\epsilon\right)=0\qquad \text{in }\mathcal{D}'(\omega),
\end{equation*}
and the limit average velocity field satisfies the ``mass-conservation'' equation
\begin{equation*}
\frac{\partial}{\partial x}\left(h\,\overline{u}_1^*\right)+\frac{\partial}{\partial y}\left(h\,\overline{u}_2^*\right)=0\qquad \text{in }\mathcal{D}'(\omega).
\end{equation*}
\end{theorem}

Furthermore, regarding strong convergence the authors \cite{chambat1986} found the next result:
\begin{theorem}
Under the hypothesis of Theorem \ref{theo:t2}, suppose there exists a function $\hat{g}\in H^{1/2}(\Gamma)$ that does not depends on $\epsilon$, such that
\begin{equation}
g^\epsilon(x,y,z)=\hat{g}(x,y,z/\epsilon)\label{eq:reynolds_limit_cond5.1}
\end{equation}
then, it holds
\begin{itemize}
\item $\epsilon^2p^\epsilon$, $\epsilon^2\frac{\partial p^\epsilon}{\partial x}$, $\epsilon^2\frac{\partial p^\epsilon}{\partial y}$ and $\epsilon \frac{\partial p^\epsilon}{\partial Z}$ converge strongly in $L^2(\omega)$ to $p^*$, $\frac{\partial p^
*}{\partial x}$, $\frac{\partial p^
*}{\partial y}$ and $0$ resp.
\item $p^*$ is unique and lies in $H^1(\omega)$, also it satisfies
$$\nabla\cdot \left(\frac{h^3}{12\mu}\nabla p^*\right)=\frac{S}{2}\,\frac{\partial h}{\partial x},$$
which corresponds to Reynolds equation \eqref*{reynoldseq} in the steady case with $U_L=S$, $U_H=V_H=V_L=0$.
%\item If, in addition the boundary $\partial \omega$ is of class $C^2$ then $\mathbf{U}^*$ in $(L^2( \Omega))^3$ is such that
%\begin{align*}
%\mathbf{\hat{U}}^\epsilon \rightarrow \mathbf{U}^*,\qquad \frac{\partial \mathbf{\hat{U}}^\epsilon}{\partial Z} \rightarrow \frac{\partial \mathbf{U}^*}{\partial Z},\qquad \epsilon \frac{\partial \mathbf{\hat{U}^\epsilon}}{\partial x} \rightarrow 0, \qquad \epsilon \frac{\partial \mathbf{\hat{U}^\epsilon}}{\partial y} \rightarrow 0,
%\end{align*}
%strongly in $(L^2( \Omega))^3$.
\end{itemize}
\end{theorem}

\subsection*{Conclusions}
\begin{itemize}
\item Reynolds Equation is an approximation of the Stokes system when $\epsilon$ is small.
\item The authors have shown that the solution of Stokes equations converges to the solution of Reynolds equation when $\epsilon$ goes to 0.
\item $h\in C^1(\overline{\omega})$ is a strong hypothesis. It would be interesting to extend this work under more \emph{realistic}  hypothesis like $h\in \lpom{\infty}$. This kind of functions can be found when considering discontinuous textures \cite{ausas07,tomanik2013,gherca2014}.
\end{itemize}

\section{Weak formulation for Reynolds equation}\label{sec:weak_form_reynolds}
Here we consider the non-dimensional velocity $S$ as $S=1$. From a classical point of view, solving Reynolds equation consists in seek for a pressure field $p\in C^2(\overline{\omega})$ satisfying the non-dimensional Reynolds equation
\begin{align}
\frac{\partial}{\partial x}\left({h^3}\frac{\partial p}{\partial x}\right)+\frac{\partial}{\partial y}\left({h^3}\frac{\partial p}{\partial y}\right)&=\frac{\partial h}{\partial x}+2\,\parder{h}{t}&&\text{ in }\omega\label{eq:reynolds_math}\\
p&=0&&\text{ in }\partial \omega,\label{eq:reynolds_math_bound}
\end{align}
where $\omega$ is a domain in $\mathbb{R}^2$ of class $C^1$ and $h$ is continuously differentiable both in space and time.

Please notice that in \eqref{reynolds_math} time is only a parameter. In the analysis we will show it will remain being a parameter.

Frequently, these hypotheses about the smoothness of $p$, $h$ and $\partial \omega$ are too strong. For instance, there are several works (both numerical and experimental) where textured surfaces are described by $h$ being discontinuous \cite{zhang2012,yin2012,shen2013,checo2014a}. For handling this, we need to look beyond the classical definition of derivative: here is where the tools of Functional Analysis appear. First, we rewrite the problem below for accomplishing weaker hypothesis. For this, first we multiply Reynolds equation \eqref*{reynolds_math} by some \emph{test function} $\phi\in\hzerooneom$ and make use of Green's formula (see \eqref{greens_formula}) to obtain
\begin{align}
\int_\omega h^3\,\nabla p \nabla \phi \,dA&=- \int_\omega \phi \frac{\partial h }{\partial x} -2 \int_\omega \phi\frac{\partial h }{\partial t}\,dA \nonumber\\
&=\int_\omega h\, \parder{\phi}{x}\,dA-2 \int_\omega\phi \frac{\partial h }{\partial t}\,dA \qquad \forall \phi\in \hzerooneom,\label{eq:reynolds_integral0}
\end{align}
with $dA=dx\,dy$. Observe the boundary term is null since $\phi=0$ a.e. in $\partial\omega$. 

Now, take a gap function $h:\omega\times[0,+\infty)\rightarrow \mathbb{R}^+$ such that
\begin{equation}
h(\cdot,t)\in \lpom{\infty}\,\forall t\in [0,+\infty)
\text{ and }\parder{h(\cdot,t)}{t}\in H^{-1}(\omega)\,\forall t\in [0,+\infty),\label{eq:reynolds_h_reg}
\end{equation}
so, as $\omega$ has finite measure, we have $h(\cdot,t)\in \lpom{p}$ for any $p \in [1,\infty]$ (see Lemma \ref{lemma:lp1}).

With this, given $\omega$ locally Lipschitz with measure $\mu(\omega)<\infty$ and $h$ accomplishing \eqref*{reynolds_h_reg}, we can rewrite our original problem as: find a function $p(\cdot,t)\in \hzerooneom$ such that
\begin{align}
\int_\omega h^3\,\nabla p \nabla \phi \,dA=\int_\omega h\, \parder{\phi}{x}\,dA-2 \int_\omega \phi\frac{\partial h }{\partial t}\,dA \qquad \forall t\in [0,\infty) \,\forall \phi\in \hzerooneom.\label{eq:reynolds_integral}
\end{align}
Now on, we use the norm on $\hzerooneom$ given by $$\|\phi\|_\hzerooneom=\|\nabla \phi\|_\lpom{2},$$
and let us assume $h$ is such that:
\begin{equation}
\text{there exist }a,b\in \mathbb{R}^+\text{ such that }0<a\leq h(x,y,t)\leq b\text{ a.e. on }\omega\,\forall t\in [0,+\infty).\label{eq:hyp_h_reynolds}
\end{equation}
Also, define the bilinear form $B(h):\hzerooneom\times\hzerooneom\rightarrow \mathbb{R}$ as 
\begin{equation}
B(h;u,v)=\int_\omega h^3\nabla u\nabla v\,dA.\label{eq:def_B}
\end{equation}
Since $h(\cdot,t)\in \lpom{2}$ and $\parder{h(\cdot,t)}{t}\in H^{-1}(\omega)$, the functional $\ell(h):\hzerooneom \rightarrow\mathbb{R}$ defined by 
\begin{equation}
\ell (h;\phi)=\int_\omega h\,\parder{\phi}{x}\,dA-2 \int_\omega \frac{\partial h }{\partial t}\phi\,dA
\label{eq:def_ell}
\end{equation}
is a linear functional on $\hzerooneom$.
\begin{proposition}Suppose $h$ satisfying \eqref*{reynolds_h_reg,hyp_h_reynolds}, then
$B$, defined in \eqref*{def_B}, is a continuous coercive bilinear form on $\hzerooneom$ and $\ell$, defined in \eqref*{def_ell}, is a continuous linear functional on $\hzerooneom$.
\begin{proof}
Bilinearity of $B$ and linearity of $\ell$ are trivial from the linearity of the operators involved. To prove continuity of $B(h)$, using Cauchy-Schwarz inequality we have
\begin{align}
B(h;u,v)=\int_\omega h^3\nabla u \nabla v\,dx&\leq b^3\,\|\nabla u\|_\lpom{2} \|\nabla v\|_\lpom{2}=b^3\,\|  u\|_\hzerooneom \|v\|_\hzerooneom.\nonumber
\end{align}
For proving coercivity, we write
\begin{align}
B(h;v,v)=\int_\omega h^3| \nabla v|^2\,dx&\geq a^3\, \|\nabla v\|^2_\lpom{2}=a^3\, \|v\|^2_\hzerooneom.\label{eq:coer_c3}
\end{align}

Now, for $\ell$, from Cauchy-Schwarz inequality we have
\begin{align}
|\ell(h;\phi)|&=\left|\int_\omega h\,\partial_x\phi\,dA-2 \int_\omega \partial_t h\,\phi\,dA\right|\\&\leq\|h(\cdot,t)\|_\lpom{2} \left\|\partial_x\phi\right\|_\lpom{2}+2\,\left\|\partial_t h(\cdot,t)\right\|_{H^{-1}(\omega)}\|\phi\|_{H^1(\omega)}\label{eq:est_l0}\\&\leq C(h(\cdot,t),\omega) \|\phi\|_\hzerooneom,\label{eq:est_l}
\end{align}
being $C(h(\cdot,t),\omega)=\|h(\cdot,t)\|_\lpom{2} +C_1(\omega)\left\|\partial_t h(\cdot,t)\right\|_{H^{-1}(\omega)}$, and $C_1$ is a Poincaré constant. Therefore, $\ell(h;\cdot)$ is continuous on $\hzerooneom$.
\end{proof}
\end{proposition}
With all this, \eqref{reynolds_integral} can be written as, for each time $t$
\begin{equation}
B(h;p,\phi)=\ell(h;\phi),\qquad\forall \phi \in \hzerooneom,\label{eq:reynolds_weak_formulation}
\end{equation}
and by definition, we say a function $p\in C^1\left(0,+\infty,C^2(\overline{\omega})\right)$ accomplishing \eqref{reynolds_math,reynolds_math_bound} is a \emph{classical solution} of \eqref*{reynolds_math}-\eqref*{reynolds_math_bound}. While a function $p(\cdot,t)\in\hzerooneom$ is a \emph{weak solution} of  \eqref*{reynolds_math}-\eqref*{reynolds_math_bound} if it satisfies \eqref{reynolds_weak_formulation}.

Thus, as both $B(h)$ and $\ell(h)$ satisfy the hypothesis of Lax-Milgram Theorem (see Appendix \ref{chap:appendix_maths}) we have the next result:
\begin{theorem}
The problem ``to find $p(\cdot,t)\in\hzerooneom$ accomplishing \eqref{reynolds_weak_formulation} for an arbitrary time $t\in [0,+\infty)$'' has a unique solution. 
\end{theorem}

\subsection{Stability Analysis}
Does the unique solution of \eqref{reynolds_weak_formulation} depends continuously on $h$?. A first idea is to take $\phi=p(\cdot,t)$ in \eqref{reynolds_weak_formulation} and so, using \eqref{est_l,coer_c3}, we have
\begin{equation}
\|p(\cdot,t)\|_\hzerooneom\leq \frac{1}{a^3}\left\{\|h(\cdot,t)\|_\lpom{2}+C_1(\omega)\|\partial_t h(\cdot,t)\|_{H^{-1}(\omega)}\right\}.\label{eq:bound0}
\end{equation}
Suppose $h$ appears only in R.H.S. of \eqref{reynolds_math} and so the functional $\ell$ depends on $h$, while $B$ does not. This would be a typical case where, due to the linearity of the equation, \eqref*{bound0} is enough to assure stability of $p$ with respect to \emph{small changes} on $h$. However, as $h$ appears on the L.H.S. of \eqref{reynolds_math}, and so the bilinear form $B$ also depends on $h$ (thus, we had written $B(h)$), stability of the solution with respect to $h$ requires some major development.

The next result is based in a similar analysis that can be found in \cite{bonito2013}.

To relax notation, for $f:\omega\times [0,+\infty)$ such that $f(\cdot,t)$ is in some normed space $X$, we denote
$$\left|f\right|_X=\|f(\cdot,\,t)\|_X.$$
\begin{theorem}
Suppose $p_1(\cdot,t),\,p_2(\cdot,t)\in\hzerooneom$ accomplish the weak formulations of Reynolds equation for an arbitrary time $t\geq 0$:
\begin{align}
\int_\omega h_1^3\nabla p_1 \nabla \phi\,dA&=\int_\omega h_1\,\parder{\phi}{x}\,dA-2 \int_\omega \frac{\partial h_1 }{\partial t}\phi\,dA\qquad \forall \phi\in \hzerooneom,\label{eq:var_u1}\\
\int_\omega h_2^3\nabla p_2 \nabla \phi\,dA&=\int_\omega h_2\,\parder{\phi}{x}\,dA-2 \int_\omega \frac{\partial h_2 }{\partial t}\phi\,dA\qquad \forall \phi\in \hzerooneom.\label{eq:var_u2}
\end{align}
Suppose also that both $h_1$ and $h_2$, satisfying \eqref*{reynolds_h_reg}, are such that $0<a_1\leq h_1(\cdot,t) \leq b_1$, $0<a_2\leq h_2(\cdot,t) \leq b_2$ a.e. on $\omega$ with 
$$\left|h_1-h_2\right|_\lpom{\infty}<\epsilon,$$
and $\partial_t h_1$, $\partial_t h_2\in {H^{-1}(\omega)}$ are such that $$\left|\partial_t h_1-\partial_t h_2\right|_{H^{-1}(\omega)}<\epsilon'.$$ Then, the next estimate holds
\begin{align}
\|p_1(\cdot,\,t)-p_2(\cdot,\,t)\|_\hzerooneom \leq\frac{1}{a_2^3} \left\{\epsilon\, \left[\frac{C(b_1+b_2,\,\omega)}{a_1^3}\left(b_1+\left|\partial_t h_1\right|_\lpom{2}\right)+C(\omega)\right]+\epsilon'\,C(\omega)\right\},\label{eq:bound4}
\end{align}
where $C(\cdot)$ are constants not depending on $h_1(\cdot,\,t)-h_2(\cdot,\,t)$ nor $\partial_t h_1(\cdot,\,t)-\partial_t h_2(\cdot,\,t)$.
\begin{proof}
Subtracting \eqref{var_u2} from \eqref{var_u1}, rearranging terms and recalling the definition of $\ell(h)$ (with $h=h_1-h_2$), for any $\phi\in \hzerooneom$ we have
$$\int_\omega h_2^3\,\nabla (p_1-p_2)\nabla \phi\,dA=-\int_\omega (h_1^3-h_2^3)\nabla p_1\nabla \phi\,dA+\ell(h_1-h_2;\phi),$$
taking $\phi=p_1-p_2$ this can be written as 
\begin{align*}
\int_\omega h_2^3\,\nabla (p_1-p_2)^2\,dA=-\int_\omega (h_1^3-h_2^3)\nabla p_1\nabla(p_1-p_2)\,dA+\ell(h;p_1-p_2),
\end{align*}
taking absolute value
\begin{align}
a_2^3\,|p_1-p_2|_\hzerooneom ^2&\leq |h_1^3-h_2^3|_\lpom{\infty}\,\left|\int_\omega \nabla p_1\nabla(p_1-p_2)\, dA\right| + |\ell(h;p_1-p_2)|\nonumber\\
&\leq |h_1^3-h_2^3|_\lpom{\infty} |p_1|_\hzerooneom|p_1-p_2|_\hzerooneom + |\ell(h;p_1-p_2)|.\label{eq:bound1}
\end{align}
By \eqref{est_l0} we have that
\begin{align*}
|\ell(h;p_1-p_2)|\leq \left(|h_1-h_2|_\lpom{2}+C_1(\omega)\,|\partial_t(h_1-h_2)|_{H^{-1}(\omega)}\right)|p_1-p_2|_\hzerooneom,
\end{align*}
replacing this in \eqref{bound1} we obtain
\begin{align}
a_2^3\,|p_1-p_2|_\hzerooneom\leq|h_1^3-h_2^3|_\lpom{\infty}|p_1|_\hzerooneom +|h_1-h_2|_\lpom{2}+C_1(\omega)|\partial_t(h_1-h_2)|_{H^{-1}(\omega)}.\label{eq:bound2}
\end{align}
The estimate
$$|h_1-h_2|_\lpom{2}\leq \mu(\omega)^\frac{1}{2}|h_1-h_2|_\lpom{\infty},$$ and 
\begin{align*}
|h_1^3-h_2^3|_\lpom{\infty}&\leq |h_1^2+h_1h_2+h_2^2|_\lpom{\infty}|h_1-h_2|_\lpom{\infty}\\&\leq C(b_1+b_2,\,\omega)\,|h_1-h_2|_\lpom{\infty},
\end{align*}
allow us to rewrite \eqref*{bound2} as
\begin{align}
a_2^3\,|p_1-p_2|_\hzerooneom \leq \epsilon\, \left(C(b_1+b_2,\,\omega)|p_1|_\hzerooneom +C(\omega)\right)+\epsilon'\,C_1(\omega).\label{eq:bound3}
\end{align}
Now, we use the estimate \eqref*{bound0} for $p_1$, so we get
$$|p_1|_\hzerooneom \leq \frac{C(\omega)}{a_1^3}\left(b_1+|\partial_t h_1|_\lpom{2}\right).$$Finally, putting the last inequality in \eqref{bound3} we obtain the result.
\end{proof}
\end{theorem}

\subsection{Spatial regularity}
From Sobolev Imbeddings (see  \Secref{sobolev_imbeddings}, and Lemma \ref{lemma:lp1}) we have that, for the two dimensional case ($n=2$), the solution $p$ is such that $p(\cdot,t)\in L^q(\omega)$ $\forall q\in [2,\infty)$. For the case $n=1$, we have an analogous weak formulation with analogous results, but this time $p(\cdot,t)\in C_B^0(\omega)$. Moreover, by Theorem \ref{theo:reg_equa} we have that if $h(\cdot,t)\in \lpOm{\infty}$, $\partial_t h(\cdot,t)\in 
\lpOm{p}$ and considering the hypothesis of $\Omega$ having finite measure and with Lipschitz boundary, if $u\in \hzerooneOm$ is a weak solution of \eqref{reynolds_math,reynolds_math_bound} then $p\in C^{0,\alpha}(\bar\Omega)$, and 
$$
\|p\|_{ C^{0,\alpha}(\bar\Omega)}\leq C\left(|\partial_t h|_{L^{p}(\Omega)}+|h|_{L^{2p}(\Omega)}\right).
$$
where the constant $C$ depends only on $n,p,\alpha,\Omega$ and $h$.
\begin{remark}
\it The hypothesis made for $h$ in \eqref*{hyp_h_reynolds} give us a huge freedom for treating much more complex surfaces than those considered on the classical formulation.
\end{remark}
\begin{remark}
\it Galerkin's Methods is a robust family of methods for solving  variational problems as the one presented in this section. Those methods use the rich structure of $\hzerooneom$ as a Hilbert space whose elements can be approximated by smooth functions. We recommend \cite{brenner2002} for an approach to that theory.
\end{remark}

\section{Maximum Principle for Reynolds equation} The Maximum Principle is an important feature of elliptic PDEs that distinguishes them from equations of higher order and systems of equations. In order to establish it, we define first a notion of inequality at the boundary for functions in the Sobolev Space $\honeom$. Let us say that $u\in \honeom$ satisfies $u\leq 0$ on the boundary $\partial \omega$ if its positive part $u^+=\max\{u,0\}\in \hzerooneom$, which is equivalent to $\left.u^+\right|_{\partial\omega}=0$ (see the properties of the trace operator in Appendix \ref{chap:appendix_maths}). If $u$ is continuous in a neighborhood of $\partial \omega$, then $u$ satisfies $u\leq 0$ on $\partial \omega$ if the inequality holds in the classical pointwise sense. We say that $u\geq 0$ on $\partial \omega$ if $-u\leq 0$ on $\partial \omega$, and $u\leq v$ (both in $\honeom$) on $\partial \omega$ if $u-v\leq 0$ on $\partial \omega$.
\begin{theorem}\label{theo:max_princ_reynolds}
Let $p\in \honeom$ satisfy $$\nabla\cdot \left(h^3\,\nabla p\right)\leq 0\left(\geq 0\right),\qquad \text{in }\omega,$$
in the weak sense, where $h$ satisfies \eqref*{hyp_h_reynolds}. Then
$$\inf_{\omega} p \geq \inf_{\partial \omega} p^-,\qquad \left(\sup_{\omega} p \leq \sup_{\partial \omega} p^+\right).$$
\end{theorem}
The proof of this theorem can be found in Chapter 8 of \cite{trudinger1983} and it is based on the boundedness of $h$ and the ellipticity of the equation. 

As an example, stationary Reynolds equation can be written as $$\nabla \cdot \left( h^3\nabla p \right) = \partial_x h,$$
including the condition $p=0$ in $\partial \omega$ and supposing  the geometry is convergent (divergent) everywhere, i.e., $\partial_x h\leq 0$ ($\geq 0$) on $\omega$, the maximum Principle establishes that $p$ must be non-negative (non-positive) over all $\omega$.

\Figref{ex_conv_div_geo} shows two instances of 1D lubrication on the domain $\omega=[0,1]$ with boundary conditions $p(0)=p(1)=0$. The first geometry have bounds between $y=0$ and the linear represented by the continuous red line. Thus, as the velocity is assumed to be positive ($S=1$), the first geometry corresponds to a convergent geometry. Its corresponding pressure profile is represented by the dashed red line. The pressure profile is non-negative as the Maximum Principle establishes. On the other hand, a divergent geometry and its non-positive pressure profile are represented by the blue line and the dashed blue line resp.
 \begin{figure}[h!]
 \centering 
 \def\svgwidth{\textwidth}	\footnotesize{
\input{figs/ex_conv_div_geo.pdf_tex}}
\caption[Disc pad scheme and pressure profile.]{Disc pad scheme and pressure profile. $p_c$ ($p_d$) is the pressure corresponding to the convergent (divergent) gap.}\label{fig:ex_conv_div_geo}
\end{figure}

Moreover, there is a stronger result 
\begin{theorem}\label{theo:strong_max_princ_reynolds}
Let $p\in C^2(\omega)\cap C^0(\overline{\omega})$, $h\in C^1(\overline{\omega})$ and
$$\nabla\cdot \left(h^3\,\nabla p\right)\leq 0\left(\geq 0\right),\qquad \text{in }\omega,$$
in the classical sense on $\omega$, which is of class $C^1$. Then if $u$ achieves its minimum (maximum) at the interior of $\omega$, $u$ is constant.
\end{theorem}
The proof of this theorem can be found in Section 3.2 of \cite{trudinger1983}.